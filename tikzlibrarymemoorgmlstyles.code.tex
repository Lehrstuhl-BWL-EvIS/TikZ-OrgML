\ProvidesFileRCS[v\pgfversion] $header: Code/Tests/MEMO_OrgML_Shapes/Library/tikzlibrarymemoorgmlshapes.code.tex,v 0.1 2015/12/05 18:30:00 F.Schneider Exp $

%~~~~~~~~~~~~~~~~~~~~~~~~~~~~~~~~~~~~~~~~~~~~~~~~~~~~~~~~~~~~~~~~~~~~~~~~~~~~~
%~~  __  __ ___ __  __  ___     ___           __  __ _    
%~~ |  \/  | __|  \/  |/ _ \   / _ \ _ _ __ _|  \/  | |   
%~~ | |\/| | _|| |\/| | (_) | | (_) | '_/ _` | |\/| | |__ 
%~~ |_|  |_|___|_|  |_|\___/   \___/|_| \__, |_|  |_|____|
%~~                                     |___/             
%~~  ____  _         _
%~~ / ___|| |_ _   _| | ___  ___
%~~ \___ \| __| | | | |/ _ \/ __|
%~~  ___) | |_| |_| | |  __/\__ \
%~~ |____/ \__|\__, |_|\___||___/
%~~            |___/ 
%~~
%~~ Implementierung der TikZ-Style für die Shapes der MEMO OrgML 
%~~ Notationssymbole 
%~~ 
%~~ Fabian Schneider (fabian.schneider@studium.fernuni-hagen.de)
%~~
%~~ ##########################################################################
%~~ Implementierung der Styles
%~~ ##########################################################################
%~~~~~~~~~~~~~~~~~~~~~~~~~~~~~~~~~~~~~~~~~~~~~~~~~~~~~~~~~~~~~~~~~~~~~~~~~~~~~

%~~~~~~~~~~~~~~~~~~~~~~~~~~~~~~~~~~~~~~~~~~~~~~~~~~~~~~~~~~~~~~~~~~~~~~~~~~~~~
%~~ TikZ-Styles
%~~~~~~~~~~~~~~~~~~~~~~~~~~~~~~~~~~~~~~~~~~~~~~~~~~~~~~~~~~~~~~~~~~~~~~~~~~~~~
%~~ ######################################
%~~ ALLGEMEIN
%~~ ######################################
%~~ Allgemeine Einstellungen, die für alle Shapes gelten
\tikzset{
    common/.style={
        line cap=round
    }    
}

%~~ ######################################
%~~ EREIGNISSE
%~~ ######################################

\tikzset{
    event/.style={
        common,
        draw,
        inner sep=3ex,
        color=event
    }
}

\tikzset{
    startevent/.style={
        event,
        color=startevent
    }
}

\tikzset{
    endevent/.style={
        event,
        color=endevent
    }
}

\tikzset{
    messageevent/.style={
        event
    }
}

\tikzset{
    informationchangeevent/.style={
        event
    }
}

\tikzset{
    newevent/.style={
        event
    }
}

\tikzset{
    modifiedevent/.style={
        event
    }
}

\tikzset{
    canceledevent/.style={
        event
    }
}

\tikzset{
    loadedevent/.style={
        event
    }
}

%~~ ######################################
%~~ PROZESSE
%~~ ######################################

\tikzset{
    unspecifiedprocess/.style={
        common,
        draw,
        inner sep=2.5ex,       
        text width=4em        
    }
}

\tikzset{
    decompositionprocess/.style={
        unspecifiedprocess
    }
}

%~~ ######################################
%~~ SYNCHRONISATION
%~~ ######################################

\tikzset{
    sync/.style={
        common,
        draw,
        inner sep=2.5ex,        
    }
}

\tikzset{
    andsync/.style={
        sync      
    }
}

\tikzset{
    orsync/.style={
        sync      
    }
}

%~~ ######################################
%~~ TEXTNODES
%~~ ######################################

\tikzset{
    memotext/.style={
        below,
        align=center,
        font=\footnotesize\sffamily        
    }
}

%~~ ######################################
%~~ VERBINDUNGSLINIEN
%~~ ######################################
\tikzset{
    memoline/.style={
        common,
        line width=0.75pt  
    }
}

%~~ ######################################
%~~ VERBINDUNGSSTÜCKE
%~~ ######################################

\tikzset{
    paraconn/.style={
        common
    }
}

%~~ ######################################
%~~ ITERATIONSSTART
%~~ ######################################
\tikzset{
    iterationstart/.style={
        draw,
        common,
        inner sep=0ex,
        color=iterationstart,
        minimum height=3ex,
        minimum width=3ex        
    }
}

%~~ ######################################
%~~ AUSNAHMEN
%~~ ######################################
\tikzset{
    exception/.style={
        common,
        inner sep=1.5ex
    }
}

\tikzset{
    unspecifiedexception/.style={
        exception
    }
}

%~~ ######################################
%~~ ENTSCHEIDUNGEN
%~~ ######################################
\tikzset{
    decision/.style={
        common,
        inner sep=2ex
    }
}

\tikzset{
    unspecifieddecision/.style={
        decision
    }
}

%~~ Ende des Paketes
\endinput