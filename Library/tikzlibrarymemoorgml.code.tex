\ProvidesFileRCS[v\pgfversion] $header: Code/Tests/MEMO_OrgML_Shapes/Library/tikzlibrarymemoorgml.code.tex,v 0.1 2015/12/05 18:30:00 F.Schneider Exp $

%~~~~~~~~~~~~~~~~~~~~~~~~~~~~~~~~~~~~~~~~~~~~~~~~~~~~~~~~~~~~~~~~~~~~~~~~~~~~~
%~~  __  __ ___ __  __  ___     ___           __  __ _    
%~~ |  \/  | __|  \/  |/ _ \   / _ \ _ _ __ _|  \/  | |   
%~~ | |\/| | _|| |\/| | (_) | | (_) | '_/ _` | |\/| | |__ 
%~~ |_|  |_|___|_|  |_|\___/   \___/|_| \__, |_|  |_|____|
%~~                                     |___/             
%~~  ___ _                      
%~~ / __| |_  __ _ _ __  ___ ___
%~~ \__ \ ' \/ _` | '_ \/ -_|_-<
%~~ |___/_||_\__,_| .__/\___/__/
%~~               |_|   
%~~
%~~ Implementierung der Macros zur Erstellung der Kontrollfluss- und De-
%~~ kompositionsdiagramme
%~~
%~~ Fabian Schneider (fabian.schneider@studium.fernuni-hagen.de)
%~~
%~~ ##########################################################################
%~~ Implementierung der Macros
%~~ ##########################################################################
%~~~~~~~~~~~~~~~~~~~~~~~~~~~~~~~~~~~~~~~~~~~~~~~~~~~~~~~~~~~~~~~~~~~~~~~~~~~~~

%~~~~~~~~~~~~~~~~~~~~~~~~~~~~~~~~~~~~~~~~~~~~~~~~~~~~~~~~~~~~~~~~~~~~~~~~~~~~~
%~~ Verwendete TikZ-Libraries
%~~~~~~~~~~~~~~~~~~~~~~~~~~~~~~~~~~~~~~~~~~~~~~~~~~~~~~~~~~~~~~~~~~~~~~~~~~~~~
\usetikzlibrary{memoorgmlshapes}
\usetikzlibrary{calc}
\usetikzlibrary{trees}
\usetikzlibrary{backgrounds}

\makeatletter

%~~~~~~~~~~~~~~~~~~~~~~~~~~~~~~~~~~~~~~~~~~~~~~~~~~~~~~~~~~~~~~~~~~~~~~~~~~~~~
%~~ TeX-Macros
%~~~~~~~~~~~~~~~~~~~~~~~~~~~~~~~~~~~~~~~~~~~~~~~~~~~~~~~~~~~~~~~~~~~~~~~~~~~~~

%~~
%~~ Vorlage für die Beschreibung der Macros
%~~~~~~~~~~~~~~~~~~~~~~~~~~~~~~~~~~~~~~~~~~~~~~~~~~~~~~~~~~~~~~~~~~~~~~~~~~~~~
% Macroname: 
% Funktion:  
% Parameter: 
%~~~~~~~~~~~~~~~~~~~~~~~~~~~~~~~~~~~~~~~~~~~~~~~~~~~~~~~~~~~~~~~~~~~~~~~~~~~~~

%~~~~~~~~~~~~~~~~~~~~~~~~~~~~~~~~~~~~~~~~~~~~~~~~~~~~~~~~~~~~~~~~~~~~~~~~~~~~~
% Macroname: memodebug
% Funktion:  Ausgabe von Meldungen in das Logfile
% Parameter: #1: Meldungstext
%            #2: Parameter
%			 #3: Parameter
%~~~~~~~~~~~~~~~~~~~~~~~~~~~~~~~~~~~~~~~~~~~~~~~~~~~~~~~~~~~~~~~~~~~~~~~~~~~~~
\def\memodebug#1#2#3{
    \message{<<MEMO-Debug: #1 #2 #3>>}
}

%~~ ######################################
%~~ KONTROLLFLUSSDIAGRAMM
%~~ ######################################

%~~~~~~~~~~~~~~~~~~~~~~~~~~~~~~~~~~~~~~~~~~~~~~~~~~~~~~~~~~~~~~~~~~~~~~~~~~~~~
% Macroname: memoprocess
% Funktion:  Erzeugen eines MEMO OrgML Prozesses mit dem passenden
%            Shape, an der definierten Stelle mit dem definierten Text
% Parameter: #1: Name des zu verwendenden Shapes
%            #2: Name des Nodes der erzeugt wird
%            #3: x-Position
%            #4: y-Position
%            #5: <OrgUnit>
%            #6: -ID-
%            #7: Bezeichnung für den Prozess
%~~~~~~~~~~~~~~~~~~~~~~~~~~~~~~~~~~~~~~~~~~~~~~~~~~~~~~~~~~~~~~~~~~~~~~~~~~~~~
\def\memoprocess#1#2#3#4#5#6#7{
    %~~ Erzeugen des Nodes der das Notationssymbol darstellt.
    %~~ Die Verwendung des richtigen Styles wird durch die Namensgleichheit
    %~~ des Styles und des Shapes erreicht.
    %~~ Die Beschriftung des Notationssymbols erfolgt durch ein Label.
    \node[#1process, shape=#1process] (#2) at (#3, #4) [label={[memotext]above:#5}] [label={[memotext]below:#6\\#7}] {};       
}

%~~~~~~~~~~~~~~~~~~~~~~~~~~~~~~~~~~~~~~~~~~~~~~~~~~~~~~~~~~~~~~~~~~~~~~~~~~~~~
% Macroname: memoevent
% Funktion:  Erzeugen eines MEMO OrgML Events mit dem passenden
%            Shape, an der definierten Stelle mit dem definierten Text
% Parameter: #1: Name des zu verwendenden Shapes
%            #2: Name des Nodes der erzeugt wird
%            #3: x-Position
%            #4: y-Position
%            #5: -ID-
%            #6: Bezeichnung für das Ereignis
%~~~~~~~~~~~~~~~~~~~~~~~~~~~~~~~~~~~~~~~~~~~~~~~~~~~~~~~~~~~~~~~~~~~~~~~~~~~~~
\def\memoevent#1#2#3#4#5#6{
    %~~ Erzeugen des Nodes der das Notationssymbol darstellt.
    %~~ Die Verwendung des richtigen Styles wird durch die Namensgleichheit
    %~~ des Styles und des Shapes erreicht.
    %~~ Die Beschriftung des Notationssymbols erfolgt durch ein Label.
    \node[#1event, shape=#1event] (#2) at (#3, #4) [label={[memotext]below:#5\\#6}] {};       
}

%~~~~~~~~~~~~~~~~~~~~~~~~~~~~~~~~~~~~~~~~~~~~~~~~~~~~~~~~~~~~~~~~~~~~~~~~~~~~~
% Macroname: memoconn
% Funktion:  Gerade Verbindung von Prozessen und Ereignissen
% Parameter: #1: Linkes Notationssymbol
%            #2: Rechtes Notationssymbol
%~~~~~~~~~~~~~~~~~~~~~~~~~~~~~~~~~~~~~~~~~~~~~~~~~~~~~~~~~~~~~~~~~~~~~~~~~~~~~
\def\memoconn#1#2{
    \begin{scope}[on background layer]
        \draw[memoline] (#1.east) -- (#2.west);
    \end{scope}
}

%~~~~~~~~~~~~~~~~~~~~~~~~~~~~~~~~~~~~~~~~~~~~~~~~~~~~~~~~~~~~~~~~~~~~~~~~~~~~~
% Macroname: memoconnpara
% Funktion:  Erzeugen einer parallelen Ausführung von Prozessen
% Parameter: #1: Ausgangsereignis
%            #2: Prozess n
%~~~~~~~~~~~~~~~~~~~~~~~~~~~~~~~~~~~~~~~~~~~~~~~~~~~~~~~~~~~~~~~~~~~~~~~~~~~~~
\def\memoconnpara#1#2{
    %~~ Prüfen ob der Knoten für das Verbindungsstück schon erzeugt wurde
    \@ifundefined{pgf@sh@ns@#1_para}{
        %~~ Der Node für die Darstellung des Verbindungssymbols wird automatisch
        %~~ erzeugt. Die Positons des Nodes wird abhängig vom Vorgängersymbol
        %~~ errechnet.
        \node[paraconn, shape=paraconn] (#1_para) at ($(#1) + (1,0)$) {};       
    }{
        %~~ Der Node für das Verbindungsymbol wird nur einmal pro 
        %~~ Event erzeugt.
    }
        
    \draw[memoline] (#1.east) -- (#1_para.west);
    \draw[memoline] (#1_para.east) |- (#2.west);    
}


%~~~~~~~~~~~~~~~~~~~~~~~~~~~~~~~~~~~~~~~~~~~~~~~~~~~~~~~~~~~~~~~~~~~~~~~~~~~~~
% Macroname: memosync
% Funktion:  Erzeugen eines Synchronisationssymbols (AND oder OR)
% Parameter: #1: Name des zu verwendenden Shapes
%            #2: Name des Nodes der erzeugt wird
%            #3: x-Position
%            #4: y-Position
%~~~~~~~~~~~~~~~~~~~~~~~~~~~~~~~~~~~~~~~~~~~~~~~~~~~~~~~~~~~~~~~~~~~~~~~~~~~~~
\def\memosync#1#2#3#4{
    %~~ Erzeugen des Nodes der das Notationssymbol darstellt.
    %~~ Die Verwendung des richtigen Styles wird durch die Namensgleichheit
    %~~ des Styles und des Shapes erreicht.
    %~~ Die Beschriftung des Notationssymbols erfolgt durch ein Label.
    \node[#1sync, shape=#1sync] (#2) at (#3, #4) {}; 
} 

%~~~~~~~~~~~~~~~~~~~~~~~~~~~~~~~~~~~~~~~~~~~~~~~~~~~~~~~~~~~~~~~~~~~~~~~~~~~~~
% Macroname: memoconnsync
% Funktion:  Synchronisation der parallel ausgeführten Prozesse
% Parameter: #1: Erster Prozess
%            #2: Verknüpfungsoperation (Symbol)
%            #3: Anchor (north, west, south, east)
%~~~~~~~~~~~~~~~~~~~~~~~~~~~~~~~~~~~~~~~~~~~~~~~~~~~~~~~~~~~~~~~~~~~~~~~~~~~~~
\def\memoconnsync#1#2#3{
    \draw[memoline] (#1) -| (#2.#3);
}

%~~~~~~~~~~~~~~~~~~~~~~~~~~~~~~~~~~~~~~~~~~~~~~~~~~~~~~~~~~~~~~~~~~~~~~~~~~~~~
% Macroname: memoiterationuntil 
% Funktion: Erzeugen einer Loop-Until Iteration  
% Parameter:  #1: Event nach dem die Iteration beginnt
%             #2: Event vor dem die Iteration endet
%             #3: Höhe der Verbindungslinie zwischen dem Start und dem Ende
%                 der Iteration
%             #4: Abstand zum Event nach dem die Iteration beginnt
%             #5: Abstand zum Event vor dem die Iteration endet
%~~~~~~~~~~~~~~~~~~~~~~~~~~~~~~~~~~~~~~~~~~~~~~~~~~~~~~~~~~~~~~~~~~~~~~~~~~~~~
\def\memoiterationuntil#1#2#3#4#5{
    %~~ Erzeugen der beiden Notationssymbole, die die Iteration darstellen.
    %~~ Die Position der Symbole wird abhängig von den Abständen zum Start- und Endereignis,
    %~~ sowie der Linienhöhe für für die Verbindungspfeile berechnet.
    
    %~~ Grüner Block mit weißem Pfeil
    \node[shape=iterationstart, iterationstart] (#1_iterationstart) at ($(#1) + (#4,0)$) {};
    %~~ Verschlungene Pfeile und UNTIL Label
    \node[shape=iteration,inner sep=2.5ex] (#2_iterationend) at ($(#2) - (#5,0)$) [label={[draw=until, label distance=-0.9em, thick, inner xsep=0.5ex, inner ysep=0.6ex,font=\sf\small, color=until]below:UNTIL}] {};
    
    %-- Erzeugen der Iterationspfeile
    \draw[very thick, color=blue, dashed, shorten <= 6pt] (#2_iterationend) -- ++(0,#3) -- ($(#1_iterationstart) + (0,#3)$);
    \draw[very thick, -latex, color=blue, dashed] ++($(#1_iterationstart) + (0,#3)$) -- (#1_iterationstart);
}

%~~~~~~~~~~~~~~~~~~~~~~~~~~~~~~~~~~~~~~~~~~~~~~~~~~~~~~~~~~~~~~~~~~~~~~~~~~~~~
% Macroname: memoiterationuntil 
% Funktion: Erzeugen einer For-Loop Iteration  
% Parameter:  #1: Event nach dem die Iteration beginnt
%             #2: Event vor dem die Iteration endet
%             #3: Höhe der Verbindungslinie zwischen dem Start und dem Ende
%                 der Iteration
%             #4: Abstand zum Event nach dem die Iteration beginnt
%             #5: Abstand zum Event vor dem die Iteration endet
%             #6: Anzahl der Iterationen
%~~~~~~~~~~~~~~~~~~~~~~~~~~~~~~~~~~~~~~~~~~~~~~~~~~~~~~~~~~~~~~~~~~~~~~~~~~~~~
\def\memoiterationloop#1#2#3#4#5#6{
    %~~ Erzeugen der beiden Notationssymbole, die die Iteration darstellen.
    %~~ Die Position der Symbole wird abhängig von den Abständen zum Start- und Endereignis,
    %~~ sowie der Linienhöhe für für die Verbindungspfeile berechnet.
    
    %~~ Grüner Block mit weißem Pfeil
    \node[shape=iterationstart, iterationstart] (#1_iterationstart) at ($(#1) + (#4,0)$) {};
    %~~ Verschlungene Pfeile und UNTIL Label
    \node[shape=iteration,inner sep=2.5ex] (#2_iterationend) at ($(#2) - (#5,0)$) [label={[draw=timesborder, label distance=-0.9em, thick, inner xsep=0.5ex, inner ysep=0.6ex,font=\sf\small]below:#6}] {};
    
    %-- Erzeugen der Iterationspfeile
    \draw[very thick, color=blue, dashed, shorten <= 6pt] (#2_iterationend) -- ++(0,#3) -- ($(#1_iterationstart) + (0,#3)$);
    \draw[very thick, -latex, color=blue, dashed] ++($(#1_iterationstart) + (0,#3)$) -- (#1_iterationstart);
}



%~~ ######################################
%~~ DEKOMPOSITIONSDIAGRAMM
%~~ ######################################

%~~~~~~~~~~~~~~~~~~~~~~~~~~~~~~~~~~~~~~~~~~~~~~~~~~~~~~~~~~~~~~~~~~~~~~~~~~~~~
% Macroname: memodecomproot
% Funktion:  Wurzelknoten für ein Dekompositionsdiagramm
% Parameter: #1: Name des zu verwendenden Shapes
%			 #2: Name des Nodes der erzeugt wird
%			 #3: x-Position
%			 #4: y-Position
%			 #5: - ID -
%~~~~~~~~~~~~~~~~~~~~~~~~~~~~~~~~~~~~~~~~~~~~~~~~~~~~~~~~~~~~~~~~~~~~~~~~~~~~~
\def\memodecomproot#1#2#3#4#5{
	\node[shape=#1,#1] (#2) at (#3,#4) [label={[memotext]below:#5}] {}
}

%~~~~~~~~~~~~~~~~~~~~~~~~~~~~~~~~~~~~~~~~~~~~~~~~~~~~~~~~~~~~~~~~~~~~~~~~~~~~~
% Macroname: memdecompchild
% Funktion:  Erzeugen eines Kindknoten in einem Dekompositionsdiagramm
% Parameter: #1: Name des zu verwendenden Shapes
%			 #2: <OrgUnit>
%			 #3: - ID -
%			 #4: Name
%~~~~~~~~~~~~~~~~~~~~~~~~~~~~~~~~~~~~~~~~~~~~~~~~~~~~~~~~~~~~~~~~~~~~~~~~~~~~~
\def\memodecompchild#1#2#3#4{
	node[shape=#1, #1] [label={[memotext]below:#3\\#4}] [label={[memotext]above:#2}] {}
}

\makeatother
%~~ Ende des Erweiterungspaketes
\endinput
