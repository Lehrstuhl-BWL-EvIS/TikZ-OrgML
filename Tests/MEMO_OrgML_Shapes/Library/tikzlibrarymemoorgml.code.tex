\ProvidesFileRCS[v\pgfversion] $header: Code/Tests/MEMO_OrgML_Shapes/Library/tikzlibrarymemoorgml.code.tex,v 0.1 2015/12/05 18:30:00 F.Schneider Exp $

%~~~~~~~~~~~~~~~~~~~~~~~~~~~~~~~~~~~~~~~~~~~~~~~~~~~~~~~~~~~~~~~~~~~~~~~~~~~~~
%~~  __  __ ___ __  __  ___     ___           __  __ _    
%~~ |  \/  | __|  \/  |/ _ \   / _ \ _ _ __ _|  \/  | |   
%~~ | |\/| | _|| |\/| | (_) | | (_) | '_/ _` | |\/| | |__ 
%~~ |_|  |_|___|_|  |_|\___/   \___/|_| \__, |_|  |_|____|
%~~                                     |___/             
%~~  ___ _                      
%~~ / __| |_  __ _ _ __  ___ ___
%~~ \__ \ ' \/ _` | '_ \/ -_|_-<
%~~ |___/_||_\__,_| .__/\___/__/
%~~               |_|   
%~~
%~~ Implementierung der Macros zur Erstellung der Kontrollfluss- und De-
%~~ kompositionsdiagramme
%~~
%~~ ##########################################################################
%~~ Implementierung der Macros
%~~ ##########################################################################
%~~~~~~~~~~~~~~~~~~~~~~~~~~~~~~~~~~~~~~~~~~~~~~~~~~~~~~~~~~~~~~~~~~~~~~~~~~~~~

%~~~~~~~~~~~~~~~~~~~~~~~~~~~~~~~~~~~~~~~~~~~~~~~~~~~~~~~~~~~~~~~~~~~~~~~~~~~~~
%~~ Verwendete TikZ-Libraries
%~~~~~~~~~~~~~~~~~~~~~~~~~~~~~~~~~~~~~~~~~~~~~~~~~~~~~~~~~~~~~~~~~~~~~~~~~~~~~
\usetikzlibrary{memoorgmlshapes}
\usetikzlibrary{calc}
\usetikzlibrary{trees}


%~~~~~~~~~~~~~~~~~~~~~~~~~~~~~~~~~~~~~~~~~~~~~~~~~~~~~~~~~~~~~~~~~~~~~~~~~~~~~
%~~ TeX-Macros
%~~~~~~~~~~~~~~~~~~~~~~~~~~~~~~~~~~~~~~~~~~~~~~~~~~~~~~~~~~~~~~~~~~~~~~~~~~~~~

%~~
%~~ Vorlage für die Beschreibung der Macros
%~~~~~~~~~~~~~~~~~~~~~~~~~~~~~~~~~~~~~~~~~~~~~~~~~~~~~~~~~~~~~~~~~~~~~~~~~~~~~
% Macroname: 
% Funktion:  
% Parameter: 
%~~~~~~~~~~~~~~~~~~~~~~~~~~~~~~~~~~~~~~~~~~~~~~~~~~~~~~~~~~~~~~~~~~~~~~~~~~~~~

%~~~~~~~~~~~~~~~~~~~~~~~~~~~~~~~~~~~~~~~~~~~~~~~~~~~~~~~~~~~~~~~~~~~~~~~~~~~~~
% Macroname: memodebug
% Funktion:  Ausgabe von Meldungen in das Logfile
% Parameter: #1: Meldungstext
%            #2: Parameter
%			 #3: Parameter
%~~~~~~~~~~~~~~~~~~~~~~~~~~~~~~~~~~~~~~~~~~~~~~~~~~~~~~~~~~~~~~~~~~~~~~~~~~~~~
\def\memodebug#1#2#3{
    \message{<<MEMO-Debug: #1 #2 #3>>}
}

%~~ ######################################
%~~ KONTROLLFLUSSDIAGRAMM
%~~ ######################################

%~~~~~~~~~~~~~~~~~~~~~~~~~~~~~~~~~~~~~~~~~~~~~~~~~~~~~~~~~~~~~~~~~~~~~~~~~~~~~
% Macroname: memonode
% Funktion:  Erzeugen eines MEMO OrgML Notationssymbols mit dem passenden
%            Shape, an der definierten Stelle mit dem definierten Text
% Parameter: #1: Name des zu verwendenden Shapes
%            #2: Name des Nodes der erzeugt wird
%            #3: x-Position
%            #4: y-Position
%            #5: Beschreibung des Ereignisses/Prozesses
%~~~~~~~~~~~~~~~~~~~~~~~~~~~~~~~~~~~~~~~~~~~~~~~~~~~~~~~~~~~~~~~~~~~~~~~~~~~~~
\def\memonode#1#2#3#4#5{
    %~~ Erzeugen des Nodes der das Notationssymbol darstellt.
    %~~ Die Verwendung des richtigen Styles wird durch die Namensgleichheit
    %~~ des Styles und des Shapes erreicht.
    %~~ Die Beschriftung des Notationssymbols erfolgt durch ein Label.
    \node[#1, shape=#1] (#2) at (#3,#4) [label={[memotext]below:#5}] {};       
}

%~~~~~~~~~~~~~~~~~~~~~~~~~~~~~~~~~~~~~~~~~~~~~~~~~~~~~~~~~~~~~~~~~~~~~~~~~~~~~
% Macroname: memoconn
% Funktion:  Gerade Verbindung von Prozessen und Ereignissen
% Parameter: #1: Linkes Notationssymbol
%            #2: Rechtes Notationssymbol
%~~~~~~~~~~~~~~~~~~~~~~~~~~~~~~~~~~~~~~~~~~~~~~~~~~~~~~~~~~~~~~~~~~~~~~~~~~~~~
\def\memoconn#1#2{
    \draw[memoline] (#1) -- (#2);
}

%~~~~~~~~~~~~~~~~~~~~~~~~~~~~~~~~~~~~~~~~~~~~~~~~~~~~~~~~~~~~~~~~~~~~~~~~~~~~~
% Macroname: memoconnpara
% Funktion:  Erzeugen einer parallelen Ausführung von Prozessen
% Parameter: #1: Ausgangsereignis
%            #2: Erster Prozess
%            #3: Zweiter Prozess
%~~~~~~~~~~~~~~~~~~~~~~~~~~~~~~~~~~~~~~~~~~~~~~~~~~~~~~~~~~~~~~~~~~~~~~~~~~~~~
\def\memoconnpara#1#2#3{
    %~~ Der Node für die Darstellung des Verbindungssymbols wird automatisch
    %~~ erzeugt. Die Positons des Nodes wird abhängig vom Vorgängersymbol
    %~~ errechnet.
    \node[paraconn, shape=paraconn] (#1_para) at ($(#1) + (1,0)$) {};
        
    \draw[memoline] (#1) -- (#1_para);
    \draw[memoline] (#1_para) |- (#2);
    \draw[memoline] (#1_para) |- (#3);
}

%~~~~~~~~~~~~~~~~~~~~~~~~~~~~~~~~~~~~~~~~~~~~~~~~~~~~~~~~~~~~~~~~~~~~~~~~~~~~~
% Macroname: memoconnsync
% Funktion:  Synchronisation der parallel ausgeführten Prozesse
% Parameter: #1: Erster Prozess
%            #2: Zweiter Prozess
%            #3: Verknüpfungsoperation
%~~~~~~~~~~~~~~~~~~~~~~~~~~~~~~~~~~~~~~~~~~~~~~~~~~~~~~~~~~~~~~~~~~~~~~~~~~~~~
\def\memoconnsync#1#2#3{
    \draw[memoline] (#1) -| (#3);
    \draw[memoline] (#2) -| (#3);
}


%~~ ######################################
%~~ DECOMPOSITIONSDIAGRAMM
%~~ ######################################

%~~~~~~~~~~~~~~~~~~~~~~~~~~~~~~~~~~~~~~~~~~~~~~~~~~~~~~~~~~~~~~~~~~~~~~~~~~~~~
% Macroname: memodecomproot
% Funktion:  Wurzelknoten für ein Dekompositionsdiagramm
% Parameter: #1: Name des zu verwendenden Shapes
%			 #2: Name des Nodes der erzeugt wird
%			 #3: x-Position
%			 #4: y-Position
%			 #5: - ID -
%~~~~~~~~~~~~~~~~~~~~~~~~~~~~~~~~~~~~~~~~~~~~~~~~~~~~~~~~~~~~~~~~~~~~~~~~~~~~~
\def\memodecomproot#1#2#3#4#5{
	\node[shape=#1,#1] (#2) at (#3,#4) [label={[memotext]below:#5}] {}
}

%~~~~~~~~~~~~~~~~~~~~~~~~~~~~~~~~~~~~~~~~~~~~~~~~~~~~~~~~~~~~~~~~~~~~~~~~~~~~~
% Macroname: memdecompchild
% Funktion:  Erzeugen eines Kindknoten in einem Dekompositionsdiagramm
% Parameter: #1: Name des zu verwendenden Shapes
%			 #2: <OrgUnit>
%			 #3: - ID -
%			 #4: Name
%~~~~~~~~~~~~~~~~~~~~~~~~~~~~~~~~~~~~~~~~~~~~~~~~~~~~~~~~~~~~~~~~~~~~~~~~~~~~~
\def\memodecompchild#1#2#3#4{
	node[shape=#1, #1] [label={[memotext]below:#2\\#3}] [label={[memotext]above:#4}] {}
}

%~~ Ende des Erweiterungspaketes
\endinput
