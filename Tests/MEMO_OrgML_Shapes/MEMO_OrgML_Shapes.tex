\documentclass[border=20pt]{standalone}
\usepackage[utf8]{inputenc}
\usepackage[ngerman]{babel}

%~~~~~~~~~~~~~~~~~~~~~~~~~~~~~~~~~~~~~~~~~~~~~~~~~~~~~~~~~~~~~~~~~~~~~~~~~~~~~
%~~ Laden von TikZ
%~~~~~~~~~~~~~~~~~~~~~~~~~~~~~~~~~~~~~~~~~~~~~~~~~~~~~~~~~~~~~~~~~~~~~~~~~~~~~
\usepackage{tikz}

%~~ Laden des Paketes für die Darstellung der MEMO-OrgML Diagramme
\usepackage{/home/fabi/Dropbox/Studium/Fernstudium/Masterthesis/Code/Tests/MEMO_OrgML_Shapes/memoorgml}

\begin{document}
\begin{tikzpicture}
    %~~ Erzeugen und plazieren der Nodes          
    \memonode{startevent}{s}{0}{0}{Anfrage \\ genehmig};
    
    %~~ Parallele Ausführung der Prozesse Logistikkosten berechnen
    %~~ und Materialkosten berechnen
    \memonode{paraconn}{p1}{1}{0}{};    
    
    %~~ Logikstikkosten berechnen
    \memonode{unspecprocess}{l1}{3}{4}{Logistikkosten berechnen};
    \memonode{chgevent}{l2}{6}{4}{Logistikkosten \\ berechnet};
    
    %~~ Synchronisation der Lohnkosten- und Logistikkostenberechnung
    \memonode{paraconn}{p2}{8}{4}{};
    
    \memonode{unspecprocess}{l3}{10}{6}{Logistikkosten bewerten};
    \memonode{chgevent}{l4}{13}{6}{Logistikkosten \\ bewertet};
    
    \memonode{unspecprocess}{l5}{10}{2}{Lohnkosten berechnen};
    \memonode{chgevent}{l6}{13}{2}{Lohnkosten \\ berechnet};
    
    %~~ Logistikkosten berechnen und Lohnkosten berechnen 
    %~~ synchronisieren
    \memonode{andsync}{s1}{15}{4}{};
    
    %~~ Materialkosten berechnen
    \memonode{unspecprocess}{m1}{3}{-4}{Materialkosten berechnen};
    \memonode{chgevent}{m2}{6}{-4}{Materialkosten \\ berechnet};
    \memonode{unspecprocess}{m3}{10}{-4}{Lohnkosten bewerten};
    \memonode{chgevent}{m4}{13}{-4}{Materialkosten \\ bewertet};
    
    %~~ Synchronisation der Berechnung der Logistikkostenberechnung und
    %~~ der Materialkostenberechnung      
    \memonode{andsync}{s2}{17}{0}{};
    \memonode{unspecprocess}{b1}{20}{0}{Bestellung \\ zusammenstellen};
    \memonode{endevent}{e}{23}{0}{Bestellung \\ zusammengestellt};
    
     
    %~~ Erzeugen der Verbindungen der Notataionssymbole 
    \memoconn{s}{p1};
    
    \memoconnpara{p1}{l1}{m1}; 
    \memoconn{l1}{l2};
    \memoconn{l2}{p2};
    
    \memoconnpara{p2}{l3}{l5};
    \memoconn{l3}{l4};
    \memoconn{l5}{l6}
    \memoconnsync{l4}{l6}{s1};
    
    \memoconn{m1}{m2};
    \memoconn{m2}{m3};
    \memoconn{m3}{m4};    

    \memoconnsync{s1}{m4}{s2};
    \memoconn{s2}{b1};
    \memoconn{b1}{e};
    
\end{tikzpicture}
\end{document}
