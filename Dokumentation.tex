\documentclass[11pt]{article}
\usepackage[ngerman]{babel}
\usepackage[T1]{fontenc}
\usepackage[utf8]{inputenc}
\usepackage{doc}
\usepackage{listings}
\usepackage{color}

%~~~~~~~~~~~~~~~~~~~~~~~~~~~~~~~~~~~~~~~~~~~~~~~~~~~~~~~~~~~~~~~~~~~~~~~~~~~~~
%~~ Formatierung des Quellcodes
%~~~~~~~~~~~~~~~~~~~~~~~~~~~~~~~~~~~~~~~~~~~~~~~~~~~~~~~~~~~~~~~~~~~~~~~~~~~~~
\definecolor{lstlila}{RGB}{139,0,139}
\definecolor{lstcomment}{RGB}{152,152,152}

\lstdefinelanguage{lsttex}{
	morekeywords={ definecolor,
				   tikzset,
				   path,
				   usetikzlibrary,
				   usepackage,
				   svg,
				   draw,
				   pic,
				   backgroundpath,
				   foregroundpath,
				   inheritbackgroundpath,
				   beforebackgroundpath,
				   pgfdeclareshape,
				   inheritsavedanchors,
				   inheritanchorborder,
				   inheritanchor,
				   southwest,
				   northeast,
				   pgfsetcolor,
				   pgfpathmoveto,
				   pgfpoint,
				   pgf,
				   pgfusepath,
				   pgfpathclose,
				   pgfpathlineto,
				   pgfsetlinewidth,
				   pgfsetfillcolor,
				   pgflineto,
				   child,
				   def,
				   memonode,
				   node,
				   shape,
				   label,
				   tikzpicture,
				   remember,
				   picture,
				   overlay,
				   subfigure,
				   figure,
				   begin,
				   end,
				   centering,
				   textwidth,
				   at,
				   line,
				   width,
				   color,
				   style,
				   beforeforegroundpath,
				   inheritbeforebackgroundpath,
				   from,
				   cap,
				   inner,
				   sep,
				   height,
				   minimum,
				   distance,
				   sibling,
				   level,
				   edge,
				   parent,
				   fork,
				   down,
				   memoprocess,
				   memoevent,
				   memoexception,
				   memoconn,
				   memoconnpara,
				   memosync,
				   memoconnsync,
				   memoconnxor,
				   memoconnxorprobability,
				   memoxor}
				  }[keywords,tex,comments]

\lstset{%
    language=lsttex,
    captionpos=t,
    breaklines=true,
    basicstyle=\footnotesize\ttfamily,
    keywordstyle=\footnotesize\ttfamily\color{lstlila},
    commentstyle=\footnotesize\ttfamily\color{lstcomment},
    xleftmargin=25pt,
    xrightmargin=25pt,
    texcsstyle=*\footnotesize\ttfamily\color{lstlila},
    numbers=left,
    numberstyle=\ttfamily\small
}

%~~~~~~~~~~~~~~~~~~~~~~~~~~~~~~~~~~~~~~~~~~~~~~~~~~~~~~~~~~~~~~~~~~~~~~~~~~~~~
%~~ Beginn der Dokumentation
%~~~~~~~~~~~~~~~~~~~~~~~~~~~~~~~~~~~~~~~~~~~~~~~~~~~~~~~~~~~~~~~~~~~~~~~~~~~~~
\begin{document}
\begin{titlepage}
\title{\textit{PGF}/\textit{TikZ} Erweiterungspaket \texttt{memoorgml}} 
\author{Fabian Schneider (fabian.schneider@studium.fernuni-hagen.de)}   
\end{titlepage}
\maketitle

\section{Einführung}
\label{sec:Einführung}

\section{Installation}
\label{sec:Installation}

\section{Bereitgestellte Notationssymbole}
\label{sec:Notationssymbole}

\section{Makros}
\label{sec:Makros}
\DescribeMacro{\memoprocess} \meta{Shape}\meta{Name}\meta{x-Pos.}\meta{y-Pos.}\meta{Org.-Einheit}\meta{ID}\meta{Bezeichnung}\medskip

Das ist ein wirklich verrückter Test.
    \begin{lstlisting}
\memoprocess{unspecified}{p1}{0}{0}{}{}{};    
    \end{lstlisting}

    \bigskip

\DescribeMacro{\memoevent} Makro für Ereignisse.\bigskip

\DescribeMacro{\memoexception} Makro für Ausnahmen.\bigskip

\DescribeMacro{\memoconn} Makro für gerade Kanten.\bigskip

\DescribeMacro{\memoparaconn} Makro für Parallelisierung.\bigskip

\DescribeMacro{\memosync} Makro für Synchronisation.\bigskip

\DescribeMacro{\memoconnsync} Makro für Synchronisation.\bigskip

\DescribeMacro{\memoconnxor} Makro für XOR-Start.\bigskip

\DescribeMacro{\memoconnxorprobability} Makro für XOR mit Wahrscheinlichkeit.\bigskip

\DescribeMacro{\memoiterationuntil} Makro für beliebige Wiederholung.\bigskip

\DescribeMacro{\memoiterationloop} Makro für For-Schleife.\bigskip

\section{Erweiterung des Pakets}
\label{sec:Erweiterung}
\end{document}
